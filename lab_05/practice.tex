\chapter{Практические задания}

\section{Задание 1}

В листинге \ref{lst:1.lisp} представлена функция, которая уменьшает на 10 все числа из списка-аргумента, проходя по верхнему уровню.
\includelistingpretty
	{1.lisp}
	{lisp}
	{Функция, которая уменьшает на 10 все числа из списка-аргумента}

\section{Задание 2}

В листинге \ref{lst:2.lisp} представлена функция, которая получает как аргумент список чисел, а возвращает список квадратов этих чисел в том же порядке.
\includelistingpretty
	{2.lisp}
	{lisp}
	{Функция, которая возвращает список квадратов чисел}

\section{Задание 3}

В листинге \ref{lst:3.lisp} представлена функция, которая умножает на заданное число-аргумент все числа из заданного списка-аргумента, когда все элементы~--- числа или любые объекты.
\includelistingpretty
	{3.lisp}
	{lisp}
	{Функция, которая умножает на заданное число-аргумент все числа из заданного списка-аргумента}

\section{Задание 4}

В листинге \ref{lst:4.lisp} представлена функция, которая по своему списку-аргументу \texttt{lst} определяет, является ли он палиндромом (то есть равны ли \texttt{lst} и \texttt{(reverse lst)}), для одноуровнего смешанного списка.
\includelistingpretty
	{4.lisp}
	{lisp}
	{Функция, которая по своему списку-аргументу \texttt{lst} определяет, является ли он палиндромом}

\section{Задание 5}

В листинге \ref{lst:5.lisp} представлен предикат \texttt{set-equal}, который возвращает t, если два его множества-аргумента содержат одни и те же элементы, порядок которых не имеет значения.
\includelistingpretty
	{5.lisp}
	{lisp}
	{Предикат \texttt{set-equal}, который возвращает t, если два его множества-аргумента содержат одни и те же элементы}

\section{Задание 6}

В листинге \ref{lst:6.lisp} представлена функция \texttt{select-between}, которая из списка-аргумента, содержащего только числа, выбирает только те, которые расположены между двумя указанными числами~--- границами~-~аргумента и возвращает их в виде списка (упорядоченного по возрастанию).
\includelistingpretty
	{6.lisp}
	{lisp}
	{Функция \texttt{select-between}}

\section{Задание 7}

В листинге \ref{lst:7.lisp} представлена функция, вычисляющая декартово произведение двух своих списков-аргументов.
\includelistingpretty
	{7.lisp}
	{lisp}
	{Функция, вычисляющая декартово произведение}

\section{Задание 8}

Описание поведения функции \texttt{reduce}:
\begin{enumerate}
	\item \texttt{(reduce \#'+ '())} возвращает 0 в силу начального значения по умолчанию, равного 0;
	\item \texttt{(reduce \#'* '())} возвращает 1 в силу начального значения по умолчанию, равного 1.
\end{enumerate}

\section{Задание 9}

В листинге \ref{lst:9.lisp} представлена функция, которая вычисляет сумму длин всех элементов-списков списка \texttt{list-of-list}.
\includelistingpretty
	{9.lisp}
	{lisp}
	{Функция, которая вычисляет сумму длин всех элементов}
