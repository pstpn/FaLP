\chapter{Практические задания}

\section{Задание 1}

На рисунке \ref{img:task1} представлены диаграммы вычисления выражений.
\includeimage
	{task1}
	{f}
	{H}
	{0.9\textwidth}
	{Диаграммы вычисления выражений}

\section{Задание 2}

В листинге \ref{lst:2.lisp} представлена реализация функции вычисления гипотенузы прямоугольного треугольника.
\includelistingpretty
	{2.lisp}
	{lisp}
	{Функция вычисления гипотенузы прямоугольного треугольника}
	
На рисунке \ref{img:task2} представлена диаграмма вычисления выражения \ref{lst:2.lisp}.
	\includeimage
	{task2}
	{f}
	{H}
	{0.55\textwidth}
	{Диаграмма вычисления выражения \ref{lst:2.lisp}}

\section{Задание 3}

В листинге \ref{lst:3.lisp} представлены результаты вычисления выражений.
\includelistingpretty
	{3.lisp}
	{lisp}
	{Результаты вычисления выражений}

\section{Задание 4}

В листинге \ref{lst:4.lisp} представлена реализация функции \texttt{longer\_then} от двух списков-аргументов, которая возвращает Т, если первый аргумент имеет большую длину.
\includelistingpretty
	{4.lisp}
	{lisp}
	{Реализация функции longer\_then}

\section{Задание 5}

В листинге \ref{lst:5.lisp} представлены результаты вычисления выражений.
\includelistingpretty
	{5.lisp}
	{lisp}
	{Результаты вычисления выражений}

\section{Задание 6}

В листинге \ref{lst:6.lisp} представлены результаты вычисления выражений функции \texttt{mystery}.
\includelistingpretty
	{6.lisp}
	{lisp}
	{Результаты вычисления выражений функции \texttt{mystery}}
	
\section{Задание 7}
	
В листинге \ref{lst:7.lisp} представлена функция \texttt{f-to-c} и результат ее выполнения.
\includelistingpretty
	{7.lisp}
	{lisp}
	{Функция \texttt{f-to-c} и результат ее выполнения}
	
\section{Задание 8}

В листинге \ref{lst:8.lisp} представлены результаты вычисления выражений.
\includelistingpretty
	{8.lisp}
	{lisp}
	{Результаты вычисления выражений}