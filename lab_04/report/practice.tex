\chapter{Практические задания}

\section{Задание 1}

Ниже представлены отличия функций \texttt{cons}, \texttt{list} и \texttt{append}:
\begin{enumerate}
	\item \texttt{cons}~--- базовая функция, которая объединяет значения двух своих аргументов в точечную пару;
	\item \texttt{list}~--- принимает произвольное число аргументов и возвращает список, состоящий из значений аргументов. 
	\item \texttt{append}~--- не разрушающая структуру функция, которая объединяет списки и возвращает новый список, содержащий комбинированные элементы.
\end{enumerate}

В листинге \ref{lst:1.lisp} приведены результаты вычисления выражений.
\includelistingpretty
	{1.lisp}
	{lisp}
	{Результаты вычисления выражений}

\section{Задание 2}

В листинге \ref{lst:2.lisp} приведены результаты вычисления выражений.
\includelistingpretty
	{2.lisp}
	{lisp}
	{Результаты вычисления выражений}

\section{Задание 3}

В листинге \ref{lst:3.lisp} приведены два варианта функции, которая возвращает последний элемент своего списка-аргумента.
\includelistingpretty
	{3.lisp}
	{lisp}
	{Функции, которые возвращают последний элемент своего списка-аргумента}

\section{Задание 4}

В листинге \ref{lst:4.lisp} приведены два варианта функции, которая возвращает свой список аргумента без последнего элемента.
\includelistingpretty
	{4.lisp}
	{lisp}
	{Функции, которые возвращают свой список аргумента без последнего элемента}

\section{Задание 5}

В листинге \ref{lst:5.lisp} приведена функция \texttt{swap-first-last}, которая переставляет в списке-аргументе первый и последний элемент.
\includelistingpretty
	{5.lisp}
	{lisp}
	{Функция \texttt{swap-first-last}}

\section{Задание 6}

В листинге \ref{lst:6.lisp} приведена реализация простой игры в кости.
\includelistingpretty
	{6.lisp}
	{lisp}
	{Реализация простой игры в кости}

\section{Задание 7}

В листинге \ref{lst:7.lisp} приведена функция, которая по своему списку-аргументу \texttt{lst} определяет, является ли он палиндромом.
\includelistingpretty
	{7.lisp}
	{lisp}
	{Функция, которая определяет, полиндром ли список}

\clearpage

\section{Задание 8}

В листинге \ref{lst:8.lisp} приведены функции, которые обрабатывают таблицу из 4-х точечных пар: (страна . столица), и возвращают по стране~--- столицу, а по столице~--- страну.
\includelistingpretty
	{8.lisp}
	{lisp}
	{Функции, которые обрабатывают таблицу из 4-х точечных пар}

\section{Задание 9}

В листинге \ref{lst:9.lisp} приведена функция, которая умножает на заданное число-аргумент первый числовой элемент списка из заданного 3-х элементного списка-аргумента, когда:
\begin{enumerate}
	\item все элементы списка~--- числа;
	\item элементы списка~--- любые объекты.
\end{enumerate}
\includelistingpretty
	{9.lisp}
	{lisp}
	{Функция, которая умножает на заданное число-аргумент первый числовой элемент списка}
